\documentclass[12pt,twocolumn]{article}
\usepackage{a4}
\usepackage[margin=.5cm,bottom=1.5cm]{geometry}

\usepackage[utf8]{inputenc}

\usepackage{amsmath}

\usepackage{mathpartir}

\usepackage{listings}
\lstset{
  mathescape=true,
  language=[Objective]{Caml},
  basicstyle=\ttfamily,
  extendedchars=true,
  showstringspaces=false,
  aboveskip=\smallskipamount,
  % belowskip=\smallskipamount,
  columns=fullflexible,
  moredelim=**[is][\color{blue}]{/*}{*/},
  moredelim=**[is][\color{green!60!black}]{/!}{!/},
  moredelim=**[is][\color{orange}]{/(}{)/},
  moredelim=[is][\color{red}]{/[}{]/},
  xleftmargin=1em,
}
\lstset{aboveskip=0.4ex,belowskip=0.4ex}

\newcommand{\match}[2]{\mathtt{match}(#1,#2)}
\newcommand{\matches}[2]{\mathtt{matches}(#1,#2)}

\newcommand{\var}[1]{\mathtt{#1}}
\newcommand{\pK}{\mathtt{K}}
\newcommand{\any}{\mathtt{\_}}

\title{Translation validation of a pattern-matching compiler}
\author{Francesco Mecca, Gabriel Scherer}

\begin{document}
\maketitle

\begin{abstract}
We propose an algorithm for the translation validation of a pattern
matching compiler for a small subset of the OCaml pattern
matching features. Given a source program and its compiled version the
algorithm checks wheter the two are equivalent or produce a counter
example in case of a mismatch.

Our equivalence algorithm works with decision trees. Source patterns are
converted into a decision tree using matrix decomposition.
Target programs, described in a subset of the Lambda intermediate
representation of the OCaml compiler, are turned into decision trees
by applying symbolic execution.
\end{abstract}
\end{document}
