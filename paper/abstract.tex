\documentclass[12pt]{article}
\usepackage{a4}
\usepackage[margin=.5cm,bottom=1.5cm]{geometry}

\usepackage[utf8]{inputenc}

\usepackage{amsmath}

\usepackage{mathpartir}

\usepackage{listings}
\lstset{
  mathescape=true,
  language=[Objective]{Caml},
  basicstyle=\ttfamily,
  extendedchars=true,
  showstringspaces=false,
  aboveskip=\smallskipamount,
  % belowskip=\smallskipamount,
  columns=fullflexible,
  moredelim=**[is][\color{blue}]{/*}{*/},
  moredelim=**[is][\color{green!60!black}]{/!}{!/},
  moredelim=**[is][\color{orange}]{/(}{)/},
  moredelim=[is][\color{red}]{/[}{]/},
  xleftmargin=1em,
}
\lstset{aboveskip=0.4ex,belowskip=0.4ex}

\newcommand{\match}[2]{\mathtt{match}(#1,#2)}
\newcommand{\matches}[2]{\mathtt{matches}(#1,#2)}

\newcommand{\var}[1]{\mathtt{#1}}
\newcommand{\pK}{\mathtt{K}}
\newcommand{\any}{\mathtt{\_}}

\title{Translation validation of a pattern-matching compiler}
\author{Francesco Mecca, Gabriel Scherer}

\begin{document}
\maketitle

\begin{abstract}
We propose an algorithm for the translation validation of a pattern
matching compiler for a small subset of the OCaml pattern
matching features. Given a source program and its compiled version the
algorithm checks wheter the two are equivalent or produce a counter
example in case of a mismatch.

Our equivalence algorithm works with decision trees. Source patterns are
converted into a decision tree using matrix decomposition.
Target programs, described in a subset of the Lambda intermediate
representation of the OCaml compiler, are turned into decision trees
by applying symbolic execution.
\end{abstract}
\section{Translation validation}
A pattern matching compiler turns a series of pattern matching clauses
into simple control flow structures such as \texttt{if, switch}, for example:
\begin{lstlisting}
  match x with
  | [] -> (0, None)
  | x::[] -> (1, Some x)
  | _::y::_ -> (2, Some y)
\end{lstlisting}
\begin{lstlisting}
(if scrutinee
    (let (field_1 =a (field 1 scrutinee))
        (if field_1
            (let
                (field_1_1 =a (field 1 field_1)
                 x =a (field 0 field_1))
                (makeblock 0 2 (makeblock 0 x)))
            (let (y =a (field 0 scrutinee))
                (makeblock 0 1 (makeblock 0 y)))))
    [0: 0 0a])
\end{lstlisting}
%% TODO: side by side
The code in the right is in the Lambda intermediate representation of
the OCaml compiler. The Lambda representation of a program is shown by
calling the \texttt{ocamlc} compiler with \texttt{-drawlambda} flag.

The pattern matching compiler is a critical part of the compiler
in terms of correctness because any bug would result in wrong code
production rather than compilation failures.
Such bugs also are hard to catch by testing because they arise in
corner cases of complex patterns which are typically not in the
compiler test suite.
In the last five years there were (only) two bugs in the OCaml pattern
matching compiler; they were found long after they were introduced.

We have considered evolving the pattern matching compiler, either by
using a new algorithm or by incremental refactorings.
We want to verify the changed compiler to ensure that no bugs were
introduced.

One could either verify the compiler implementation or check each
input output pair. We chose the latter technique, translation
validation; it which gives a weaker result but is easier to adopt in
the case of a production compiler. The compiler is treated as a
blackbox and proof only depends on our equivalence algorithm between
source and target programs.

\section{Our approach}
%% replace common TODO
Our algorithm translates both source and target programs into a common
representation, decision trees. Here is the decision tree for the
source example program.
\begin{verbatim}
       Node(Root)
       /        \
     (= [])    (= ::)
     /             \
   Leaf         Node(Root.1)
(0, None)       /         \
             (= [])      (= ::)
             /               \
          Leaf              Leaf
   (1, Some(Root.0))   (2, Some(Root.1.0))
\end{verbatim}

Target decision trees have a similar shape but the tests on the
branches are related to the low level representation of values in
Lambda code. For example, cons cells \texttt{x::xs} are blocks with
tag 0.

To check the equivalence of a source and a target decision tree,
we proceed by case analysis.
If we have two leaves, we check that the two right-hand-sides are
equivalent.
If we have a node $N$ and another tree $T$ we check equivalence for
each child of $N$, which is a pair of a branch condition $\pi_i$ and a
subtree $C_i$. For every child $(\pi_i, C_i)$ we reduce $T$ by killing all
the branches that are incompatible with $\pi_i$ and check that the
reduced tree is equivalent to $C_i$.

\end{document}
